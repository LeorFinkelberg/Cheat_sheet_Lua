\documentclass[%
	11pt,
	a4paper,
	utf8,
	%twocolumn
		]{article}	

\usepackage{style_packages/podvoyskiy_article_extended}


\begin{document}
\title{Заметки. Практика использования и наиболее полезные конструкции языка \texttt{Lua}}

\author{\itshape Подвойский А.О.}

\date{}
\maketitle

\thispagestyle{fancy}

Здесь приводятся заметки по некоторым вопросам, касающимся программирования на языке \texttt{Lua} в контексте работы с системой компьютерной верстки \LaTeX.


%\shorttableofcontents{Краткое содержание}{1}

%\tableofcontents

\section{Общие замечания}

Lua не нужен разделитель между идущими подряд операторами, но в принципе можно использовать точку с запятой, если хочется. Обычно точку с запятой ставят только, если требуется разделить два и более операторов, записанных в одной строке. Переводы строк не играют никакой роли в синтаксисе Lua.

\begin{lstlisting}[
style = lua,
numbers = none	
]
-- определяет функция факториала
function fact(n)
  if n == 0 then
    return 1
  else
    return n * fact(n - 1)
  end
end

print("Enter a number:")
a = io.read("*n") -- считывает число
print(fact(a))
\end{lstlisting}

Для выхода из интерактивного режима и интерпретатора следует набрать управляющий символ конца файла (\textsf{Ctrl+D} в UNIX, \textsf{Ctrl+Z} в Windows) или вызвать функцию \texttt{exit} из библиотеки операционной системы -- для этого нужно набрать \texttt{os.exit()}.

Выполнять куски кода в интерактивном режиме можно с помощью функции \texttt{dofile}. Например
\begin{lstlisting}[
style = lua,
numbers = none	
]
-- lib.lua
function norm(x, y)
  return math.sqrt(x^2 + y^2)
end

function twice(x)
  return 2*x
end

-- интерактивная оболочка
> dofile("lib.lua")
> n = norm(3, 4) --> 5.0
\end{lstlisting}

Неопределенные переменные возвращают nil. Lua поддерживает однострочные комментраии (\verb|--|) и блочные многострочные \verb|--[[...--]]|
\begin{lstlisting}[
style = lua,
numbers = none	
]
--[[
  print(10) -- ничего не происходит
--]]

---[[
  print(10) --> 10
--]]
\end{lstlisting}

В первом случае обычный блочный комментарий, а во втором -- блок начинается с обычного однострочного комментария (\verb|--|), поэтому все, код выполняется.

Тип \texttt{nil} -- это тип с единственным значением, основная задача которого состоит в том, чтобы отличаться ото всех остальных значений. Lua использует nil как \emph{нечто}, \emph{не являющееся значением}, чтобы изобразить отсутствие подходящего значения.

ЗАМЕЧАНИЕ: Проверки условий считают \texttt{nil} и булево \texttt{false} ложными, а все прочие значения истинными. В частности, при проверках условий Lua считает ноль и пустую строку истинными значениями.

Тип \texttt{number} представляет \emph{вещественные числа}, т.е. числа двойной точности с плавающей точкой (тип double в C). В Lua нет целочисленного типа (тип integer в C).

У целых чисел есть точное представление и потому нет ошибок округления. 

Узнать длину строки можно с помощью \emph{оператора длины} (\verb|#|)
\begin{lstlisting}[
style = lua,
numbers = none	
]
print(#"python") --> 6
\end{lstlisting}

Для определения границ строковых литералов можно использовать как одинарные, так и двойные кавычки
\begin{lstlisting}[
style = lua,
numbers = none	
]
a = "a line"
b = 'another line'
c = [[
  multilines
]]
\end{lstlisting}

\section{Шпаргалка}

\subsection{Условия и циклы}

Цикл \texttt{while}
\begin{lstlisting}[
style = lua,
numbers = none
]
num = 42

while num < 50 do
  num = num + 1 -- составных операторов типа `+=` нет
end
\end{lstlisting}

Условия
\begin{lstlisting}[
style = lua,
numbers = none	
]
if num > 40 then
  print("over 40")
elseif s ~= "walternate" then -- ~= это `не равно`; проверка на равенство выполняется как в Python с помощью `==`
  io.write("not over 40\n") -- по умолчанию выводит в stdout
else
  -- по умолчанию переменные глобальные
  thisGlobal = 5
  -- для того чтобы сделать переменную локальной
  local line = io.read() -- читает строку из stdin
  -- строки можно склеить с помощью оператора `..`
  print("Winter is coming, " .. line)
end
\end{lstlisting}

Цикл \texttt{for}
\begin{lstlisting}[
style = lua,
numbers = none	
]
karlSum = 0
for i = 1, 100 do -- диапазон включает и левую, и правую границу
  karlSum = karlSum + i
end

-- чтобы переменная цикла принимала значения от большего к меньшему
fredSum = 0
for j = 100, 1, -1 do fredSum = fredSum + j end
\end{lstlisting}

Цикл \texttt{repeat}
\begin{lstlisting}[
style = lua,
numbers = none	
]
repeat
  print("The way of the future")
  num = num - 1
until num == 0
\end{lstlisting}

\subsection{Функции}

\begin{lstlisting}[
style = lua,
numbers = none
]
function fib(n)
  if n < 2 then
    return 1
  else
    return fib(n - 2) + fib(n - 1)
  end
end
\end{lstlisting}

Замыкания (вложенные функции) и анонимные функции
\begin{lstlisting}[
style = lua,
numbers = none	
]
function adder(x)
  -- вложенная анонимная функция создается, когда вызывается adder, и запоминает x
  return function (y) return x + y end -- анонимная функция
end

a1 = adder(9)
a2 = adder(36)

print(a1(16)) --> 25
print(a2(64)) --> 100
\end{lstlisting}

Функции могут быть локальными и глобальными
\begin{lstlisting}[
style = lua,
numbers = none
]
-- одно и тоже
function f(x) return x * x end
f = function (x) return x * x end

-- и это тоже
local function g(x) return math.sin(x) end
local g; g = function (x) return math.sin(x) end
-- 
\end{lstlisting}

\section{Таблицы}

Таблицы в Lua представляют собой ассоциативные массивы. Ключи таблиц по умолчанию имеют строковый тип
\begin{lstlisting}[
style = lua,
numbers = none	
]
t = {key1 = "value1", key2 = false} -- таблица; как словарь в Python

print(t.key1) --> "value1"
t.newKey = {} -- добавить новую пару "ключ-значение"
t.key2 = nil -- удалить ключ key2 из таблицы

-- можно передать значение в функцию динамически, на ходу создавая таблицу
function h(x)
  return x.key1
end

print(h{ key1 = "value" }) --> 'value'

-- итерации по таблице
for key, value in pairs(u) do
  print(key, value)
end
\end{lstlisting}

Использование таблиц как списков / массивов
\begin{lstlisting}[
style = lua,
numbers = none
]
v = {"value1", "value2", 1.21, "gigawatts"} -- здесь неявно ключами значений становятся целочисленные ключи

for i = 1, #v do
  print(v[i])
end
\end{lstlisting}

\section{Метатаблицы и метаметоды}

Таблица может иметь метатаблицу, с помощью которой можно перегружать операторы и тем самым менять их поведение
\begin{lstlisting}[
style = lua,
numbers = none	
]
f1 = {a = 1, b = 2}
f2 = {a = 2, b = 3}

metafraction = {}
function metafraction.__add(f1, f2)
  sum = {}
  sum.b = f1.b * f2.b
  sum.a = f1.a * f2.b + f2.a * f1.b
  return sum
end

setmetatable(f1, metafraction)
setmetatable(f2, metafraction)

s = f1 + f2 -- вызовет __add(f1, f2)
\end{lstlisting}

Таблицы можно расширять
\begin{lstlisting}[
style = lua,
numbers = none
]
defaultFavs = {animal = "gru", food = "donuts"}
myFavs = {food = "pizza"}
setmetatable(myFavs, {__index = defaultFavs})
eatenBy = myFavs.animal --> 'gru'
\end{lstlisting}

\verb|__add|, \verb|__index| и пр. называют метаметодами. Вот полный список метаметодов
\begin{itemize}
	\item \verb|__add(a, b)|: \texttt{a + b},
	
	\item \verb|__sub(a, b)|: \texttt{a - b},
	
	\item \verb|__mul(a, b)|: \texttt{a * b},
	
	\item \verb|__div(a, b)|: \texttt{a / b},
	
	\item \verb|__mod(a, b)|: \texttt{a \% b},
	
	\item \verb|__pow(a, b)|: \verb|a ^ b|,
	
	\item \verb|__unm(a)|: \verb|-a|,
	
	\item \verb|__concat(a, b)|: \verb|a .. b|,
	
	\item \verb|__len(a)|: \verb|#a|,
	
	\item \verb|__eq(a, b)|: \verb|a == b|,
	
	\item \verb|__lt(a, b)|: \verb|a < b|,
	
	\item \verb|__le(a, b)|: \verb|a <= b|,
	
	\item \verb|__index(a, b)|: \verb|a.b|,
	
	\item \verb|__newindex(a, b)|: \verb|a.b = c|,
	
	\item \verb|__call(a, ...)|: \verb|a(...)|.
\end{itemize}



%\listoffigures\addcontentsline{toc}{section}{Список иллюстраций}

% Источники в "Газовой промышленности" нумеруются по мере упоминания 
\begin{thebibliography}{99}\addcontentsline{toc}{section}{Список литературы}
	\bibitem{hostmann:scala-2013}{\emph{Иерузалимски Р.} Программирование на языке Lua, 2013. -- 413~с. }
\end{thebibliography}

\end{document}

\documentclass[%
	11pt,
	a4paper,
	utf8,
	%twocolumn
		]{article}	

\usepackage{style_packages/podvoyskiy_article_extended}


\begin{document}
\title{Заметки. Практика использования и наиболее полезные конструкции языка \texttt{Lua}}

\author{\itshape Подвойский А.О.}

\date{}
\maketitle

\thispagestyle{fancy}

Здесь приводятся заметки по некоторым вопросам, касающимся программирования на языке \texttt{Lua} в контексте работы с системой компьютерной верстки \LaTeX.


%\shorttableofcontents{Краткое содержание}{1}

\tableofcontents

\section{Начало работы}

Lua не нужен разделитель между идущими подряд операторами, но в принципе можно использовать точку с запятой, если хочется. Обычно точку с запятой ставят только, если требуется разделить два и более операторов, записанных в одной строке. Переводы строк не играют никакой роли в синтаксисе Lua.

\begin{lstlisting}[
style = lua,
numbers = none	
]
-- определяет функция факториала
function fact(n)
  if n == 0 then
    return 1
  else
    return n * fact(n - 1)
  end
end

print("Enter a number:")
a = io.read("*n") -- считывает число
print(fact(a))
\end{lstlisting}

Для выхода из интерактивного режима и интерпретатора следует набрать управляющий символ конца файла (\textsf{Ctrl+D} в UNIX, \textsf{Ctrl+Z} в Windows) или вызвать функцию \texttt{exit} из библиотеки операционной системы -- для этого нужно набрать \texttt{os.exit()}.

Выполнять куски кода в интерактивном режиме можно с помощью функции \texttt{dofile}. Например
\begin{lstlisting}[
style = lua,
numbers = none	
]
-- lib.lua
function norm(x, y)
  return math.sqrt(x^2 + y^2)
end

function twice(x)
  return 2*x
end

-- интерактивная оболочка
> dofile("lib.lua")
> n = norm(3, 4) --> 5.0
\end{lstlisting}

Lua поддерживает однострочные комментраии (\verb|--|) и блочные многострочные \verb|--[[...--]]|
\begin{lstlisting}[
style = lua,
numbers = none	
]
--[[
  print(10) -- ничего не происходит
--]]

---[[
  print(10) --> 10
--]]
\end{lstlisting}

В первом случае обычный блочный комментарий, а во втором -- блок начинается с обычного однострочного комментария (\verb|--|), поэтому все, код выполняется.



%\listoffigures\addcontentsline{toc}{section}{Список иллюстраций}

% Источники в "Газовой промышленности" нумеруются по мере упоминания 
\begin{thebibliography}{99}\addcontentsline{toc}{section}{Список литературы}
	\bibitem{hostmann:scala-2013}{\emph{Иерузалимски Р.} Программирование на языке Lua, 2013. -- 413~с. }
\end{thebibliography}

\end{document}
